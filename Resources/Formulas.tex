\documentclass[11pt]{article}
\usepackage{mathtools}
\usepackage{listings}
\usepackage{xcolor}
%\usepackage[margin=0.8in]{geometry}
\addtolength{\oddsidemargin}{-.875in}
\addtolength{\evensidemargin}{-.875in}
\addtolength{\textwidth}{1.75in}
\addtolength{\topmargin}{-1.3in}
\addtolength{\textheight}{2in}
\definecolor{codegreen}{rgb}{0,0.6,0}
\definecolor{codegray}{rgb}{0.5,0.5,0.5}
\definecolor{codepurple}{rgb}{0.58,0,0.82}
\definecolor{backcolour}{rgb}{0.95,0.95,0.92}
\lstdefinestyle{mystyle}{
    backgroundcolor=\color{backcolour},   
    commentstyle=\color{codegreen},
    keywordstyle=\color{blue},
    numberstyle=\tiny\color{codegray},
    stringstyle=\color{codepurple},
    basicstyle=\ttfamily\footnotesize,
    breakatwhitespace=false,         
    breaklines=true,                 
    captionpos=b,                    
    keepspaces=true,                 
    numbers=left,                    
    numbersep=5pt,                  
    showspaces=false,                
    showstringspaces=false,
    showtabs=false,                  
    tabsize=2
}
\lstset{style=mystyle}
\newcommand{\nCr}[2]{\,_{#1}C_{#2}}
\newcommand{\nPr}[2]{\,_{#1}P_{#2}}
%Gummi|065|=)
\title{\textbf{Fundamental formulas, theorems and snippets}}
\author{Oscar Burga}
\date{Dec 2019}
\begin{document}

\maketitle

\section{Modular Arithmetic}
	\subsection{Fermat's little theorem}
		For any integer $a$ and prime number $p$:
			\begin{itemize}
				\item $ a^p \equiv a\;(\bmod\;p)\quad\Rightarrow\quad 
						(a^p-a)$ is an integer multiple of $p$ 
				\item $ a^{p-1} \equiv 1\;(\bmod\;p)\quad\Rightarrow\quad 
						(a^{p-1}-1)$ is an integer multiple of $p$
			\end{itemize}
	\subsection {Euler's theorem: A generalization of Fermat's little theorem}
		\begin{itemize}
			\item Euler's totient function $(\phi(n))$:
					Counts the number of integers $x\in[1,n]$ which are coprime to $n$.
			\item Phi-function properties (follows from Fermat's little theorem): 
				\begin {itemize}
					\item $ \phi(p) = p-1 $ for prime $p$
					\item $ \phi(p^k)= p^k - p^{k-1}$ for prime $p$
					\item $ \phi(ab) = \phi(a)\cdot\phi(b)$ for coprime $a$ and $b$
					\item $ \phi(ab) = \phi(a)\cdot\phi(b)\cdot\frac{d}{\phi(d)}$ 
							for not coprime $a$ and $b$, where $d = gcd(a,b)$ 
					\item Number of valid $x$ such that $x\in[1,n)$ and $gcd(x, n) = k$
							can be solved as $\phi(\frac{n}{k})$
				\end {itemize}
			\item Application in Euler's Theorem:
				\begin {itemize}
					\item $x \equiv y\;(\bmod\;\phi(n))\Rightarrow a^x\equiv a^y\;(\bmod\;n)$
						for coprime $a$ and $n$
					\item $x^{\phi(m)} \equiv 1\;(\bmod\;m)$ for coprime $x$ and $m$
					\item $ x^n \equiv x^{n\bmod \phi(m)}\;(\bmod\;m) $ for coprime $x$ and $m$
					\item $x^n \equiv x^{\phi(m) + [n\bmod\phi(m)]} (\bmod\;m)$
							for arbitrary $x$, $m$ and $n\geq\log_{2}m$
				\end {itemize}
		\end{itemize}
\section{Combinatorics}
\begin{itemize}
	\item Permutation:
	$$\nPr{n}{r} = \binom{n}{r}\cdot r! = \frac{n!}{(n-r)!} $$
	\item Combination:
	$$\nCr{n}{r} = \binom{n}{r} = \frac{n!}{r!(n-r)!} $$
	$$\binom{n}{r} = \binom{n-1}{r-1}+\binom{n-1}{r} $$
	\item Snippet for Pascal's Triangle (overflow past $\binom{62}{31}$)
	\begin{lstlisting}[language=C++]
	ll dp[maxn][maxn];
	ll comb(ll n, ll r){
		if (r == 0 || n == r) return 1;
		if (dp[n][r] != -1) return dp[n][r];
		dp[n][r] = comb(n-1, r-1) + comb(n-1, r);
		return dp[n][r];
	}\end{lstlisting}
% 
\end{itemize}
\end{document}
